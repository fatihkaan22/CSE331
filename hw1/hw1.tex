% Created 2020-10-28 Wed 15:46
% Intended LaTeX compiler: pdflatex
\documentclass[a4paper,12pt]{article}
\usepackage[utf8]{inputenc}
\usepackage[T1]{fontenc}
\usepackage{graphicx}
\usepackage{grffile}
\usepackage{longtable}
\usepackage{wrapfig}
\usepackage{rotating}
\usepackage[normalem]{ulem}
\usepackage{amsmath}
\usepackage{textcomp}
\usepackage{amssymb}
\usepackage{capt-of}
\usepackage{hyperref}
\documentclass{article}
\usepackage{here}
\usepackage{xcolor}
\usepackage[margin=1.0in]{geometry}
\usepackage{amsmath}
\usepackage{parskip}
\renewcommand\arraystretch{1.4}
\documentclass[12pt]{report}
\author{Fatih Kaan Salgır - 171044009}
\date{}
\title{CSE 331 Computer Organizations Homework 1}
\hypersetup{
 pdfauthor={Fatih Kaan Salgır - 171044009},
 pdftitle={CSE 331 Computer Organizations Homework 1},
 pdfkeywords={},
 pdfsubject={},
 pdfcreator={Emacs 27.1 (Org mode 9.3.7)}, 
 pdflang={English}}
\begin{document}

\maketitle

\section*{Question 1}
\label{sec:org32f5429}
\begin{itemize}
\item Decrasing in the wafer cost, will also decrase the cost of the chip.\\
\end{itemize}

\textbf{Today}:  Wafer cost \(\rightarrow\) 10 000 \$

Wafer cost will decrease 20\% every year 

\textbf{4 years later}:  Wafer cost \(\rightarrow\) \$10000 \texttimes{} 0.8\textsuperscript{4} = 4096 \$

\quad

\begin{itemize}
\item Yiled is the percantage of good dies from the total number of dies on the wafer. Therefore decrasing in the yeild will increase the cost of the chip.
\end{itemize}

Yield will decrease 10\% every year

\textbf{4 years later}:  Yield will be \(\rightarrow\) \(0.9^4 = 0.6561\) of the inital yield, which is: \(0.6561 \times 0.8 = 0.52488\)

$$\mbox{Cost per die} = \frac{\mbox{Cost of wafer}}{\mbox{Dies per wafer} \times \mbox{yield}}$$

\begin{flalign*}
 & \mbox{ Cost per die } = \frac{4096}{120 \times 0.52488} = 65.03\ \$ &
\end{flalign*}

Cost of single chip manifacturing will be 65.03 \$ after 4 years.

\newpage

\section*{Question 2}
\label{sec:org97b336e}
\subsection*{A)}
\label{sec:org12787a0}
\[ 
\mbox{ CPU clock cycles } = \sum\limits_{i=1}^{n} ( \mbox{CPI_i} \times \mbox{C_i} )
\]


\[
\mbox{ CPU clock cycles_{compiler A}  } = 10^{6} \times ( (50 \times 2) + (10 \times 4) + (2 \times 3) ) = 146 \times 10^{6}\ cycles 
 \]

\[
\mbox{ CPU clock cycles_{compiler B}  } = 10^{6} \times ( (80 \times 2) + (5 \times 4) + (1 \times 3) ) = 183 \times 10^{6}\ cycles
 \]

Compiler A has the lower clock cycles, so it must be faster. Compiler A \(\frac{183 \times 10^6}{146 \times 10^6} = 1.25\) times faster than compiler B.

\subsection*{B)}
\label{sec:org85a3161}

\begin{align*} 
  \mbox{ Execution time } &= \mbox{ number of cycles } \times \mbox{ clock cycle time } & \\
  \mbox{ Clock cycle time } &=  \frac{\mbox{execution time}}{\mbox{number of cycles}} &
\end{align*} 

\begin{flalign*} 
  \mbox{ Clock cycle time } = \frac{100 \times 10^{-3} }{146 \times 10^{6}} &= \frac{100}{146} \times 10^{-9} \ seconds & \\
  \mbox { Clock rate } = \frac{1}{\frac{100}{146} \times 10^{-9}} &= 1.46\  \mbox{GHz}
\end{flalign*} 
\end{document}